\documentclass[landscape]{article}

\usepackage[spanish]{babel}
\usepackage[T1]{fontenc}
\usepackage[left=2.50cm, right=2.50cm]{geometry}

\usepackage[utf8x]{inputenc}
\usepackage{amsmath}
\usepackage{amsfonts}
\usepackage{amssymb}
\usepackage{graphicx}
\usepackage{multicol}
\usepackage{changepage}
\usepackage{float}
\usepackage{cite}
\usepackage{url}

\usepackage{fancyhdr}
\pagestyle{fancy}

\usepackage{xcolor}
\usepackage{listings}
\lstset{basicstyle=\ttfamily,
  showstringspaces=false,
  commentstyle=\color{red},
  keywordstyle=\color{blue}
}


\author{Christian Barzalobre}

%\title{Portada siempre practica}

\fancyhf{}
\fancyhead[LO]{\leftmark} % En las páginas impares, parte izquierda del encabezado, aparecerá el nombre de capítulo
\fancyhead[RE]{\rightmark} % En las páginas pares, parte derecha del encabezado, aparecerá el nombre de sección
\fancyhead[RO,LE]{\thepage} % Números de página en las esquinas de los encabezados
\fancyfoot[LE,RO]{https://github.com/ChristianBarzalobre/Manual-Git.git} %Escribo este texto a la izquierda en las páginas impares y a la derecha en las pares
\renewcommand{\footrulewidth}{0.0pt}
\renewcommand{\headrulewidth}{0.0pt}

\begin{document}

\lstset{language=bash}  
\textbf{Comandos Git} \\

\begin{table}[ht]
\begin{center}
\begin{tabular}{| l | l |}

\hline
Propósito & Comando\\ \hline

Instalación & 
\begin{tabular}{@{}l@{}}
\begin{lstlisting}
pacman -S git , apt isntall git , yum install git ,
dnf isntall git ,zypper install git
\end{lstlisting}\end{tabular}\\ \hline

Comprobar instalación & 
\begin{lstlisting}
git --version
\end{lstlisting}\\ \hline


Registrar nombre de usuario  & 
\begin{lstlisting}
git config --global user.name "Nombre de usuario deseado" 
\end{lstlisting}\\ \hline

Registrar tu correo electrónico &
\begin{lstlisting}
git config --global user.email tucorreo@electronico.com
\end{lstlisting}\\ \hline

Carpeta a iniciar el git &
\begin{lstlisting}
git init
\end{lstlisting}\\ \hline

Status del git &
\begin{lstlisting}
git status
\end{lstlisting}\\ \hline

Agregar archivo al rastreo de git &
\begin{lstlisting}
git add ejemplo.cpp
\end{lstlisting}\\ \hline

Añadir una version al repositorio &
\begin{lstlisting}
git commit -m "comentario sobre la version"
\end{lstlisting}\\ \hline

Ver historial de versiones &
\begin{lstlisting}
git log --oneline
\end{lstlisting}\\ \hline

Regresar version especifica &
\begin{lstlisting}
git checkout #codigo
\end{lstlisting}\\ \hline

Regresar a la ultima version &
\begin{lstlisting}
git checkout master
\end{lstlisting}\\ \hline

Informacion detallada de las versiones &
\begin{lstlisting}
git log
\end{lstlisting}\\ \hline

Muestra los cambios realizados entre versiones &
\begin{lstlisting}
git show
\end{lstlisting}\\ \hline

Agregar todos los archivos del directorio &
\begin{lstlisting}
git add .
\end{lstlisting}\\ \hline

Eliminar versiones anteriores permanentemente &
\begin{lstlisting}
git reset #codigo --hard
\end{lstlisting}\\ \hline

Eliminar versiones anteriores sin ser tan agresivo &
\begin{lstlisting}
git reset #codigo --soft
\end{lstlisting}\\ \hline

Ver ramas &
\begin{lstlisting}
git branch
\end{lstlisting}\\ \hline

Crear una nueva rama &
\begin{lstlisting}
git branch nombredelarama
\end{lstlisting}\\ \hline

Cambiar de rama &
\begin{lstlisting}
git checkout nombredelarama
\end{lstlisting}\\ \hline

\begin{tabular}{@{}l@{}}
Trasladar cambios de una rama a otra \\ primero posicionarse en la rama donde \\ se trasladaran los cambios, \\luego trasladar los cambios
\end{tabular}  &
\begin{lstlisting}
git checkout nombredelarama , git merge ramaaexportar
\end{lstlisting}\\ \hline

\end{tabular}
\end{center}
\end{table}

\begin{table}[ht]
\begin{center}
\begin{tabular}{| l | l |}
\hline

Agrega y comenta los cambios &
\begin{lstlisting}
git commit -am "comentario sobre la version"
\end{lstlisting}\\ \hline
git ls-files
Agregar conexión a repositorio remoto &
\begin{lstlisting}
git remote add origin https://github.com/user-name/Nombre-Repositorio.git
\end{lstlisting}\\ \hline

Comprobar el repositorio remoto &
\begin{lstlisting}
git remote -v
\end{lstlisting}\\ \hline

Subir archivos al repositorio &
\begin{lstlisting}
git push -u origin nombredelarama
\end{lstlisting}\\ \hline

Clonar repositorio remoto &
\begin{lstlisting}
git clone https://github.com/user-name/Nombre-Repositorio.git
\end{lstlisting}\\ \hline

\begin{tabular}{@{}l@{}}
Clonar repositorio cambiando \\nombre de la carpeta
\end{tabular} &
\begin{lstlisting}
git clone https://github.com/**/***.git nombredelacarpeta
\end{lstlisting}\\ \hline

Actualizar la version del repositorio remoto &
\begin{lstlisting}
git pull 
\end{lstlisting}\\ \hline

Elimnar archivo del git &
\begin{lstlisting}
git rm tuarchivo.ejemplo
\end{lstlisting}\\ \hline

Listar archivos contenidos en el git &
\begin{lstlisting}
git ls-files
\end{lstlisting}\\ \hline

\end{tabular}
\end{center}
\end{table}


\end{document}
