\documentclass{report}

\usepackage[spanish]{babel}
%\usepackage[T1]{fontenc}
\usepackage[left=2.50cm, right=2.50cm]{geometry}

\usepackage[utf8x]{inputenc}
\usepackage{amsmath}
\usepackage{amsfonts}
\usepackage{amssymb}
\usepackage{graphicx}
\usepackage{multicol}
\usepackage{changepage}
\usepackage{float}
\usepackage{cite}
\usepackage{url}

\usepackage{xcolor}
\usepackage{listings}
\lstset{basicstyle=\ttfamily,
  showstringspaces=false,
  commentstyle=\color{red},
  keywordstyle=\color{blue}
}

\author{Nombre}

\title{Portada siempre practica}

\begin{document}	
\lstset{language=bash}  
\textbf{Comandos Git} \\

\begin{table}[ht]
\begin{center}
\begin{tabular}{| l | l |}

\hline
Propósito & Comando\\ \hline

Instalación & 
\begin{lstlisting}
pacman -S git
\end{lstlisting}\\ \hline

Comprobar instalación & 
\begin{lstlisting}
git --version
\end{lstlisting}\\ \hline


Registrar nombre de usuario  & 
\begin{lstlisting}
git config --global user.name "Nombre de usuario deseado" 
\end{lstlisting}\\ \hline

Registrar tu correo electrónico &
\begin{lstlisting}
git config --global user.email tucorreo@electronico.com
\end{lstlisting}\\ \hline

Carpeta a iniciar el git &
\begin{lstlisting}
git init
\end{lstlisting}\\ \hline

Status del git &
\begin{lstlisting}
git status
\end{lstlisting}\\ \hline

Agregar archivo al rastreo de git &
\begin{lstlisting}
git add ejemplo.cpp
\end{lstlisting}\\ \hline

Añadir una version al repositorio &
\begin{lstlisting}
git commit -m "comentario sobre la version"
\end{lstlisting}\\ \hline

Ver historial de versiones &
\begin{lstlisting}
git log --oneline
\end{lstlisting}\\ \hline



\end{tabular}
\end{center}
\end{table}
 
\end{document}